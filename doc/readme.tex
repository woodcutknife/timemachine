\documentclass[a4paper]{article}

\newcommand{\Code}[1]{{\large\tt #1}}

\author{Di Wang, 1300012802}

\begin{document}

\title{linux time machine}
\maketitle

\tableofcontents

\section{Introduction}
\textbf{backup} is a file backup tool under Linux, it can be used to backup and restore data locally or remotely. It has those features:
\begin{itemize}
    \item Incremental backup.
    \item Configuration with a tiny file.
    \item Easy to set excluded files and folders.
    \item Pretty display of backup status.
    \item Can limit the number of backup versions.
    \item Can restore data to a specified version.
    \item Remote backup support - use SSH.
\end{itemize}
And this documentation is generated with the help of {\large\tt htlatex}.

\section{Installation}
\textbf{backup} use {\large\tt autoconf} to manage configuration and installation. So to install just enter:\\
\\
{\tt
    \$ aclocal \\
    \$ autoconf \\
    \$ automake --add-missing \\
    \$ ./configure \\
}
\\
If there's no error during configuration, then enter:\\
\\
{\tt
    \$ make install \\
}
\\
The \textbf{backup} then has been installed to {\large\tt /usr/local/bin/backup}, and the documentation has been installed to {\large\tt /usr/local/share/doc/backup/readme.html}. \\
\\
To uninstall, just enter: \\
\\
{\tt
    \$ make uninstall \\
}
\\
Hint: perhaps you'll need to {\large\tt sudo}.

\section{Configuration}
The config file is defaultly located at {\large\tt /usr/local/etc/backup.conf}, but you can modify this in {\large\tt /usr/local/bin/backup} as you want.

\subsection{{\large\tt BACKUP}}
To specify globally what files and folders you want to backup. Remember to insert a {\large\tt DELIMITER} between two path.

\subsection{{\large\tt DELIMITER}}
To split {\large\tt BACKUP}.

\subsection{{\large\tt TARGET}}
To specify where to place your backups. It could be a path on your remote server.

\subsection{{\large\tt BACKUP\_UPPER\_LIMIT}}
To configure the upper limit of the number of versions \textbf{backup} should save.

\subsection{{\large\tt SSH}}
To switch mode from backup locally and backup remotely. \textbf{Remember that if you want to use ssh, it's supposed to connect passwordlessly}.

\subsection{{\large\tt SSH\_USER}}
To specify the ssh user name.

\subsection{{\large\tt SSH\_HOST}}
To specify the ssh host name(or ip).

\subsection{{\large\tt DAILY\_BACKUP\_HOUR}}
To specify the hour of daily backup time.

\subsection{{\large\tt DAILY\_BACKUP\_MINUTE}}
To specify the minute of daily backup time.

\section{Usage}
The general command line usage is: \\
\\
{\tt
    backup \{config,status,push,restore,daily,help\}
}

\subsection{{\large\tt backup config}}
This command simply tells where the config file locates.

\subsection{{\large\tt backup status}}
This command is used to check backup status. \\
\\
{\tt
    backup status [file-or-folder]
} \\
\\
If a file or a folder name is specified, \textbf{backup} will check the backup for this file or folder to find all available versions. If not, \textbf{backup} will check status of all the files and folders specified in config file.

\subsection{{\large\tt backup push}}
This command is used to backup data(push them!). \\
\\
{\tt
    backup push [file-or-folder]
} \\
\\
Just like the {\large\tt status} command, if a file of a folder name is specified, \textbf{backup} will push this file or folder to your backups. If not, \textbf{backup} will push all the files and folders specified in config file.

\subsubsection{{\large\tt .backupignore}}
Just like {\large\tt .gitignore}, \textbf{backup} support user-defined {\large\tt .backupignore} files. \\
Wherever you want to ignore something, just put the \textbf{relative} path in a file named {\large\tt .backupignore}. \\
\\
For example, now you're at a folder named {\large\tt test}, and you want to ignore {\large\tt test/file-to-ignore, just create {\large\tt .backupignore} in this folder, add: \\
\\
{\tt
    file-to-ignore \\
}

\subsection{{\large\tt backup restore}}
This command is used to restore data to a version of the data. \\
\\
{\tt
    backup restore file-or-folder version
} \\
\\
There is an example: \\
\\
{\tt
    backup restore ./test.sh 2014-05-25T15\_13\_18
}

\subsection{{\large\tt backup daily}}
This command is used to start or stop daily backup. The backup time is configured in the config file. \\
\\
{\tt
    backup daily \{start,stop\}
}

\subsection{{\large\tt backup help}}
This command tells you how to use \textbf{backup}, though it's a little bit ugly. \\
\\
{\tt
    backup help \{config,status,push,restore,daily,help\}
}

\end{document}
